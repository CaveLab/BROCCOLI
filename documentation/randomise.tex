\section{Randomise}

BROCCOLI provides a function for permutation testing at the group level, similar to the FSL function randomise. In its simplest form, a permutation test can with the bash wrapper be performed as

\begin{verbatim}
RandomiseGroupLevel volumes.nii -design design.mat -contrasts design.con 
\end{verbatim}
where volumes.nii is a 4D file with data from all subjects in the group(s). The design.mat file can for example be defined as

\begin{verbatim}
NumRegressors 2
NumSubjects 10

1.0 0.0
1.0 0.0
1.0 0.0
1.0 0.0
1.0 0.0
0.0 1.0
0.0 1.0
0.0 1.0
0.0 1.0
0.0 1.0
\end{verbatim}

while the design.con file can be defined as

\begin{verbatim}
NumRegressors 2
NumContrasts 2

1.0 -1.0
-1.0 1.0
\end{verbatim}


\section{OpenCL options}

The following OpenCL options are available

\begin{itemize}

\item -platform
\newline \newline The OpenCL platform to use (default 0).

\item -device
\newline \newline The OpenCL device to use (default 0).

\end{itemize}

\section{Permutation options}

The following permutation options are available

\begin{itemize}

\item -design
\newline \newline The design matrix to apply in each permutation. 

\item -contrasts                 
\newline \newline The contrast vector(s) to apply to the estimated beta values. 

\item -groupmean                 
\newline \newline Test for group mean, using sign flipping \newline (design and contrast not needed). 

\item -mask                      
\newline \newline A mask that defines which voxels to permute (default none). 

\item -permutations              
\newline \newline Number of permutations to use (default 10,000). 

\item -teststatistics            
\newline \newline Test statistics to use, 0 = GLM t-test, 1 = GLM F-test, 2 = CCA, 3 = Searchlight (default 0). Only t-test is currently implemented. 

\item -inferencemode             
\newline \newline Inference mode to use, 0 = voxel, 1 = cluster extent, 2 = cluster mass, 3 = threshold free cluster enhancement (TFCE) (default 1). 

\item -cdt                       
\newline \newline Cluster defining threshold for cluster inference (default 2.5). 

\item -significance              
\newline \newline The significance level to calculate the threshold for (default 0.05). 

\item -permutationfile 
\newline \newline Use a specific permutation file or sign flipping file (e.g. from FSL), \newline provide the filename after the option.

\end{itemize}

\section{Outputs}

By default, the function saves the results as volumes\_perm\_tvalues.nii (t-scores for the unpermuted case) and volumes\_perm\_pvalues.nii (p-values from the permutation test).

\section{Output options}

The following output options are available

\begin{itemize}

\item -output 
\newline \newline Set output filename (default volumes\_perm\_tvalues.nii and volumes\_perm\_pvalues.nii).

\item -writepermutationvalues    
\newline \newline Write all the permutation values to a text file, \newline provide the filename after the option.

\item -writepermutations         
\newline \newline Write all the random permutations (or sign flips) to a text file, \newline provide the filename after the option.

\end{itemize}

\newpage

\section{Additional options}

The following additional options are available

\begin{itemize}

\item -quiet 
\newline \newline Don't print anything to the terminal (default false). 

\item -verbose
\newline \newline Print extra stuff (default false). 

\end{itemize}


