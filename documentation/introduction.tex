\section{Introduction}

BROCCOLI is a software mainly created for analysis of functional magnetic resonance imaging (fMRI) data. The major advantage of BROCCOLI, compared to other software packages, is that it is much faster. The main reason for this is that BROCCOLI is written in OpenCL (Open Computing Language), making it possible to run the analysis in parallel on a large variety of hardware platforms (such as CPUs, Nvidia GPUs and AMD GPUs). 

\section{Downloading BROCCOLI}

BROCCOLI is available as open source at https://github.com/wanderine/BROCCOLI/. For Linux and Mac computers, the following git command can be used to download BROCCOLI into the directory BROCCOLI. 

\begin{verbatim}
git clone https://github.com/wanderine/BROCCOLI.git BROCCOLI
\end{verbatim}
For Windows computers, see https://windows.github.com/ or download BROCCOLI as a zip file from github. 

\section{Installing OpenCL}

BROCCOLI requires that at least one OpenCL driver is installed. All hardware platforms (e.g. Intel, AMD and Nvidia) require a specific OpenCL driver. It is possible to install several OpenCL drivers on a single machine. \\ \\ Intel drivers (runtime) are currently located at \\ \\ https://software.intel.com/en-us/articles/opencl-drivers , \\ \\ and can for Linux also be downloaded from \\ \\ https://dl.dropboxusercontent.com/u/4494604/opencl\_runtime\_15.1\_x64\_5.0.0.57.tar
\\ \\ AMD drivers can be found at http://developer.amd.com/tools-and-sdks/opencl-zone/opencl-tools-sdks/amd-accelerated-parallel-processing-app-sdk/ \\ \\ Nvidia drivers can be found at http://www.nvidia.com/Download/index.aspx. Note that a driver for an Nvidia graphics cards includes an OpenCL driver. 

\section{Checking the OpenCL drivers}

To check the installation of the OpenCL driver(s) on Linux and Mac computers, the BROCCOLI bash function GetOpenCLInfo can be used (located in the directory BROCCOLI/compiled/Bash/Linux/Release and BROCCOLI/compiled/Bash/Mac/Release). It lists all the available devices for each OpenCL platform. The following output is from a computer with three OpenCL drivers (Intel, AMD, Nvidia) and one Intel CPU (with 32 GB of memory), one AMD graphics card (3 GB of memory) and one Nvidia graphics card (6 GB of memory). For each bash function, it is easy to select which platform and which device to use. Note that the Intel CPU can be used with the Intel platform or the AMD platform. 

\begin{verbatim}

[andek@localhost]$ ./GetOpenCLInfo 
Device info 
 
---------------------------------------------
Platform number: 0
---------------------------------------------
Platform vendor: Intel(R) Corporation
Platform name: Intel(R) OpenCL
Platform extentions: cl_khr_fp64 cl_khr_icd cl_khr_global_int32_base_atomics 
cl_khr_global_int32_extended_atomics cl_khr_local_int32_base_atomics 
cl_khr_local_int32_extended_atomics cl_khr_byte_addressable_store 
cl_intel_printf cl_ext_device_fission cl_intel_exec_by_local_thread 
Platform profile: FULL_PROFILE
---------------------------------------------

---------------------------------------------
Device number: 0
---------------------------------------------
Device vendor: Intel(R) Corporation
Device name:        Intel(R) Core(TM) i7-3770K CPU @ 3.50GHz
Hardware version: OpenCL 1.2 (Build 67279)
Software version: 1.2
OpenCL C version: OpenCL C 1.2 
Device extensions: cl_khr_fp64 cl_khr_icd cl_khr_global_int32_base_atomics 
cl_khr_global_int32_extended_atomics cl_khr_local_int32_base_atomics 
cl_khr_local_int32_extended_atomics cl_khr_byte_addressable_store 
cl_intel_printf cl_ext_device_fission cl_intel_exec_by_local_thread 
Global memory size in MB: 32071
Global memory cache size in KB: 256
Local memory size in KB: 32
Constant memory size in KB: 128
Parallel compute units: 8
Clock frequency in MHz: 3500
Max number of threads per block: 1024
Max number of threads in each dimension: 1024 1024 1024

---------------------------------------------
Platform number: 1
---------------------------------------------
Platform vendor: Advanced Micro Devices, Inc.
Platform name: AMD Accelerated Parallel Processing
Platform extentions: cl_khr_icd cl_amd_event_callback cl_amd_offline_devices
Platform profile: FULL_PROFILE
---------------------------------------------

---------------------------------------------
Device number: 0
---------------------------------------------
Device vendor: Advanced Micro Devices, Inc.
Device name: Tahiti
Hardware version: OpenCL 1.2 AMD-APP (1214.3)
Software version: 1214.3 (VM)
OpenCL C version: OpenCL C 1.2 
Device extensions: cl_khr_fp64 cl_amd_fp64 cl_khr_global_int32_base_atomics 
cl_khr_global_int32_extended_atomics cl_khr_local_int32_base_atomics 
cl_khr_local_int32_extended_atomics cl_khr_int64_base_atomics 
cl_khr_int64_extended_atomics cl_khr_3d_image_writes 
cl_khr_byte_addressable_store cl_khr_gl_sharing cl_ext_atomic_counters_32 
cl_amd_device_attribute_query cl_amd_vec3 cl_amd_printf 
cl_amd_media_ops cl_amd_media_ops2 cl_amd_popcnt cl_khr_image2d_from_buffer 
Global memory size in MB: 3035
Global memory cache size in KB: 16
Local memory size in KB: 32
Constant memory size in KB: 64
Parallel compute units: 32
Clock frequency in MHz: 1000
Max number of threads per block: 256
Max number of threads in each dimension: 256 256 256

---------------------------------------------
Device number: 1
---------------------------------------------
Device vendor: GenuineIntel
Device name: Intel(R) Core(TM) i7-3770K CPU @ 3.50GHz
Hardware version: OpenCL 1.2 AMD-APP (1214.3)
Software version: 1214.3 (sse2,avx)
OpenCL C version: OpenCL C 1.2 
Device extensions: cl_khr_fp64 cl_amd_fp64 cl_khr_global_int32_base_atomics 
cl_khr_global_int32_extended_atomics cl_khr_local_int32_base_atomics 
cl_khr_local_int32_extended_atomics cl_khr_int64_base_atomics 
cl_khr_int64_extended_atomics cl_khr_3d_image_writes 
cl_khr_byte_addressable_store cl_khr_gl_sharing cl_ext_device_fission 
cl_amd_device_attribute_query cl_amd_vec3 cl_amd_printf cl_amd_media_ops 
cl_amd_media_ops2 cl_amd_popcnt 
Global memory size in MB: 32071
Global memory cache size in KB: 32
Local memory size in KB: 32
Constant memory size in KB: 64
Parallel compute units: 8
Clock frequency in MHz: 1600
Max number of threads per block: 1024
Max number of threads in each dimension: 1024 1024 1024

---------------------------------------------
Platform number: 2
---------------------------------------------
Platform vendor: NVIDIA Corporation
Platform name: NVIDIA CUDA
Platform extentions: cl_khr_byte_addressable_store cl_khr_icd 
cl_khr_gl_sharing cl_nv_compiler_options cl_nv_device_attribute_query 
cl_nv_pragma_unroll 
Platform profile: FULL_PROFILE
---------------------------------------------

---------------------------------------------
Device number: 0
---------------------------------------------
Device vendor: NVIDIA Corporation
Device name: GeForce GTX TITAN
Hardware version: OpenCL 1.1 CUDA
Software version: 331.67
OpenCL C version: OpenCL C 1.1 
Device extensions: cl_khr_byte_addressable_store cl_khr_icd cl_khr_gl_sharing 
cl_nv_compiler_options cl_nv_device_attribute_query cl_nv_pragma_unroll  
cl_khr_global_int32_base_atomics cl_khr_global_int32_extended_atomics 
cl_khr_local_int32_base_atomics cl_khr_local_int32_extended_atomics cl_khr_fp64 
Global memory size in MB: 6143
Global memory cache size in KB: 224
Local memory size in KB: 48
Constant memory size in KB: 64
Parallel compute units: 14
Clock frequency in MHz: 875
Max number of threads per block: 1024
Max number of threads in each dimension: 1024 1024 64

\end{verbatim}

\section{Setting environment variables}

Just like for FSL, it is necessary to set a Linux environment variable called BROCCOLI\_DIR. The variable should contain the path of the BROCCOLI installation. Using bash, it can for example be set by

\begin{verbatim}
export BROCCOLI_DIR=/home/andek/BROCCOLI/
\end{verbatim}
Note that the path should end with a "/" character. To make the change permanent, add the expression to your configuration file (e.g. /home/andek/.bash\_profile). \\ \\ It may also be a good idea to add the BROCCOLI path to your local PATH variable, such that the BROCCOLI functions can be called from any location, e.g. for Linux

\begin{verbatim}
export PATH=$PATH:/home/andek/BROCCOLI/compiled/Bash/Linux/Release
\end{verbatim}
and for Mac

\begin{verbatim}
export PATH=$PATH:/home/andek/BROCCOLI/compiled/Bash/Mac/Release
\end{verbatim}
Some parts of BROCCOLI support OpenMP (open multi processing on the CPU), to speedup code not written in OpenCL. The number of OpenMP threads can, for example, be set using

\begin{verbatim}
export OMP_NUM_THREADS=4
\end{verbatim}
to run 4 threads in parallel. If this variable is not defined, the default is to use a single CPU thread.

\section{Selecting the OpenCL platform and device}

As mentioned previously, it is easy to select which OpenCL platform and device to use for each BROCCOLI function (the default platform and device is 0). For each bash function, the option -platform sets the OpenCL platform to use, and the option -device sets the device to use. The first level analysis can, for example, be performed using platform 1 and device 1 by using the command

\begin{verbatim}
./FirstLevelAnalysis fMRI.nii T1.nii BrainTemplate.nii ...
 regressors.txt contrasts.txt -platform 1 -device 1
\end{verbatim}
For the test computer used in the introduction, this means that the analysis runs on the Intel CPU using the AMD platform. By changing the device to 0, the AMD graphics card will instead be used. 

\section{Compiling the OpenCL kernel code}

The files kernelBayesian.cpp, kernelClusterize.cpp, kernelConvolution.cpp, kernelMisc.cpp, kernelRegistration.cpp, kernelStatistics1.cpp, kernelStatistics2.cpp and kernelWhitening.cpp (in the directory BROCCOLI/code/Kernels) contain all the OpenCL code that runs on the selected device. The first time BROCCOLI uses a specific device, it has to compile all the OpenCL code to a form that can be used by the device. For this reason, the first function call will take more time. The compiled code will, however, be saved as binary files (for Linux and Mac computers in the BROCCOLI/compiled/Kernels directory), such that BROCCOLI can simply read the compiled code for all subsequent function calls. For our specific test computer, BROCCOLI will for example produce the following binary files for the Nvidia device

\begin{verbatim}
broccoli_lib_kernel_Nvidia_GeForceGTXTITAN_Bayesian.bin
broccoli_lib_kernel_Nvidia_GeForceGTXTITAN_Clusterize.bin 
broccoli_lib_kernel_Nvidia_GeForceGTXTITAN_Convolution.bin
broccoli_lib_kernel_Nvidia_GeForceGTXTITAN_Misc.bin 
broccoli_lib_kernel_Nvidia_GeForceGTXTITAN_Registration.bin
broccoli_lib_kernel_Nvidia_GeForceGTXTITAN_Statistics1.bin
broccoli_lib_kernel_Nvidia_GeForceGTXTITAN_Statistics2.bin
broccoli_lib_kernel_Nvidia_GeForceGTXTITAN_Whitening.bin
\end{verbatim}
If compilation of a kernel file fails, some information will be given in the corresponding build info file (e.g. buildInfo\_Nvidia\_GeForceGTXTITAN\_Convolution.txt) in the directory BROCCOLI/compiled/Kernels. 

\section{Compiling the BROCCOLI library}

BROCCOLI is written as a C++/OpenCL library, such that it can be linked to a number of softwares. Precompiled versions of the BROCCOLI library are located in BROCCOLI/compiled/BROCCOLI\_LIB/Linux and in BROCCOLI/compiled/BROCCOLI\_LIB/Mac . To compile the BROCCOLI library, it is necessary to first install an OpenCL SDK. For Intel, the OpenCL SDK (code builder) can be downloaded from \\ \\ https://software.intel.com/en-us/articles/opencl-drivers \\ \\ and for Linux also from \\ \\ https://dl.dropboxusercontent.com/u/4494604/intel\_code\_builder\_for\_opencl\_2015\_5.0.0.62\_x64.tar \\ \\ To compile the BROCCOLI library on a Linux computer, the following command can be used from the directory BROCCOLI/code/BROCCOLI\_LIB 

\begin{verbatim}
 ./compile_broccoli_library.sh 
\end{verbatim}
Note that you may need to edit compile\_broccoli\_library.sh to point to a directory which contains opencl.h. \\ \\ For Mac, the BROCCOLI library can be compiled by using a similar bash script

\begin{verbatim}
 ./compile_broccoli_library_mac.sh 
\end{verbatim}
In these scripts it is easy to change the compilation mode from release to debug. \\ \\ On Windows computers, the BROCCOLI library can for example be compiled using Microsoft Visual Studio. Visual Studio project files are available in BROCCOLI/code/BROCCOLI\_LIB. 

\section{Compiling the bash wrappers}

It is possible to run BROCCOLI in a number of ways, for example from a Matlab terminal or from a Python terminal. The fMRI software packages FSL and AFNI can be launched from a Linux terminal, and BROCCOLI includes a bash wrapper for this purpose. Precompiled versions of the bash functions are located in BROCCOLI/compiled/Bash/Linux and in BROCCOLI/compiled/Bash/Mac . \\ \\ \\ \\ The following bash script can be used to compile the bash wrappers (available in BROCCOLI/code/Bash\_Wrapper)

\begin{verbatim}
 ./compile_wrappers.sh 
\end{verbatim}
Similarly, the following script can be used for Mac

\begin{verbatim}
 ./compile_wrappers_mac.sh 
\end{verbatim}
In these scripts it is easy to change the compilation mode from release to debug. Since BROCCOLI uses the NIfTI library to read NIfTI files, it may be necessary to first compile the NIfTI library, by running make in the directory BROCCOLI/code/Bash\_Wrapper/nifticlib-2.0.0.

\section{Required libraries}

For Linux computers, it may be necessary to install some development libraries, to be able to compile the BROCCOLI library.

\begin{verbatim}
yum -y install libXp.x86_64

yum -y install libXpm.x86_64

yum -y install libXmu.x86_64
 
yum -y install gcc-c++
 
yum -y install zlib-devel
\end{verbatim}

\begin{verbatim}
sudo apt-get install zlib1g-dev
\end{verbatim}




