\section{Introduction}

BROCCOLI provides a separate function for motion correction. In its simplest form, motion correction can with the bash wrapper be performed as

\begin{verbatim}
MotionCorrection fMRI.nii
\end{verbatim}
where fMRI.nii is a 4D fMRI dataset.

\section{OpenCL options}

The following OpenCL options are available

\begin{itemize}

\item -platform
\newline \newline The OpenCL platform to use (default 0).

\item -device
\newline \newline The OpenCL device to use (default 0).

\end{itemize}

\section{Motion correction options}

The following motion correction options are available

\begin{itemize}

\item -referencevolume
\newline \newline Provide a reference volume to which all other volumes will be aligned to (default false). If a 4D dataset is provided, all volumes will be aligned to the first volume in this 4D dataset.

\item -iterations  
\newline \newline Number of iterations for the motion correction (default 5). 

\end{itemize}

\section{Outputs}

By default, the function saves the results as input\_mc.nii. The estimated motion parameters are saved as input\_motionparameters.1D. They can for example be viewed using the AFNI function 1dplot (1dplot input\_motionparameters.1D).

\section{Output options}

The following output options are available

\begin{itemize}

\item -output 
\newline \newline Set output filename (default input\_mc.nii). 

\end{itemize}

\section{Additional options}

The following additional options are available

\begin{itemize}

\item -quiet 
\newline \newline Don't print anything to the terminal (default false). 

\item -verbose
\newline \newline Print extra stuff (default false).
 
\item -debug 
\newline \newline Get additional debug information saved as nifti files (default no). Warning: This will use a lot of extra memory! 

\end{itemize}


