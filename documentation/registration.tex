\section{Introduction}

BROCCOLI provides a separate function for image registration. Linear as well as non-linear registration is applied to the input volume. The function automagically resizes and rescales the input volume to match the reference volume. In its simplest form, a registration between two volumes can with the bash wrapper be performed as

\begin{verbatim}
RegisterTwoVolumes input_volume.nii reference_volume.nii
\end{verbatim}

\section{OpenCL options}

The following OpenCL options are available

\begin{itemize}

\item -platform
\newline \newline The OpenCL platform to use (default 0).

\item -device
\newline \newline The OpenCL device to use (default 0).

\end{itemize}

\section{Registration options}

The following registration options are available

\begin{itemize}

\item -iterationslinear          
\newline \newline Number of iterations for the linear registration (default 10), 0 means that no linear registration is performed (the two volumes must then be of the same size).  

\newpage

\item -iterationsnonlinear       
\newline \newline Number of iterations for the non-linear registration (default 10), 0 means that no non-linear registration is performed. 

%\item -lowestscale               
%\newline \newline The lowest scale for the linear and non-linear registration, should be 1, 2, 4 or 8 (default 4), x means downsampling a factor x in each dimension. A lower scale means that a larger distance between the two volumes can be handled.

\item -sigma                    
\newline \newline Amount of Gaussian smoothing applied for regularization of the displacement field, defined as sigma of the Gaussian kernel (default 5.0).

\item -zcut                      
\newline \newline Number of mm to cut from the bottom of the input volume, can be negative, useful if the head in the volume is placed very high or low (default 0). 

\end{itemize}

\section{Outputs}

By default, the function saves the results as input\_volume\_aligned\_linear.nii (the result after linear registration) and input\_volume\_aligned\_nonlinear.nii (the result after non-linear registration). 

%\newpage 

\section{Output options}

The following output options are available

\begin{itemize}

\item -savefield                 
\newline \newline Save the estimated displacement field to file (default false). This will generate input\_volume\_displacement\_x.nii, input\_volume\_displacement\_y.nii.gz and input\_volume\_displacement\_z.nii.gz. They can be used with the function TransformVolume.

\item -saveinterpolated          
\newline \newline Save the input volume rescaled and resized to the size and resolution of the reference volume, before alignment (default false). 

\item -output 
\newline \newline Set output filename (default input\_volume\_aligned\_linear.nii and input\_volume\_aligned\_nonlinear.nii). 

\end{itemize}

\section{Additional options}

The following additional options are available

\begin{itemize}

\item -mask                      
\newline \newline Mask to apply after linear and non-linear registration, to the aligned volume. The option can for example be used to do a skullstrip (default none). 

\item -maskoriginal                      
\newline \newline Mask to apply after linear registration. The option can for example be used to do a skullstrip. Returns the volume skullstripped and unregistered (but interpolated to the reference volume size) (default none). 

\item -quiet 
\newline \newline Don't print anything to the terminal (default false). 

\item -verbose
\newline \newline Print extra stuff (default false).
 
\item -debug 
\newline \newline Get additional debug information saved as nifti files (default no). Warning: This will use a lot of extra memory! 

\end{itemize}


