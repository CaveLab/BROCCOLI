\section{Introduction}

BROCCOLI provides a separate function for transformation of volumes. In its simplest form, a non-linear transformation can with the bash wrapper be performed as

\begin{verbatim}
TransformVolume volume_to_transform.nii reference_volume.nii -field 
displacement_field_x.nii displacement_field_y.nii displacement_field_z.nii
\end{verbatim}
where reference\_volume.nii is the reference volume that was used for the registration, and displacement\_field\_x.nii displacement\_field\_y.nii displacement\_field\_z.nii are the non-linear deformation fields generated by RegisterTwoVolumes. For a linear transformation, the following command can be used

\begin{verbatim}
TransformVolume volume_to_transform.nii reference_volume.nii -matrix affinematrix.txt
\end{verbatim}
where affinematrix.txt contains a 4 x 4 affine transformation matrix (generated manually or by RegisterTwoVolumes).

\section{OpenCL options}

The following OpenCL options are available

\begin{itemize}

\item -platform
\newline \newline The OpenCL platform to use (default 0).

\item -device
\newline \newline The OpenCL device to use (default 0).

\end{itemize}

\newpage

\section{Transformation options}

The following transformation options are available

\begin{itemize}

\item -matrix
\newline \newline Apply an affine transformation matrix (4 x 4) given in a text file.

\item -scaling
\newline \newline Apply a scaling to each dimension. Using '-scaling 1' will return the volume to transform, interpolated to the size of the reference volume.

\item -centering
\newline \newline Center the volume mass of a single volume or all volumes in a 4D dataset. 

\item -field
\newline \newline Apply an arbitrary deformation field provided in three separate nifti files. The deformation field needs to have the same size as the reference volume.

\item -interpolation
\newline \newline The interpolation to use, 0 = nearest, 1 = trilinear (default 1). Nearest interpolation can for example be useful if you want to transform a mask with only integer values.

\item -zcut
\newline \newline Number of mm to cut from the bottom of the input volume, can be negative (default 0). Should be the same as for the call to RegisterTwoVolumes.

\end{itemize}

\section{Outputs}

By default, the function saves the result as volume\_to\_transform\_warped.nii. 

\section{Output options}

The following output options are available

\begin{itemize}

\item -output 
\newline \newline Set output filename (default volume\_to\_transform\_warped.nii). 

\end{itemize}

\section{Additional options}

The following additional options are available

\begin{itemize}

\item -quiet 
\newline \newline Don't print anything to the terminal (default false). 

\end{itemize}


